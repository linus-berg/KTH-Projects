\documentclass[a4paper]{scrartcl}
\usepackage[utf8]{inputenc}
\usepackage[english]{babel}
\usepackage{graphicx}
\usepackage{lastpage}
\usepackage{pgf}
\usepackage{wrapfig}
\usepackage{fancyvrb}
\usepackage{fancyhdr}
\usepackage{float}
\usepackage{hyperref}
\pagestyle{fancy}

% Create header and footer
\headheight 27pt
\pagestyle{fancyplain}
\lhead{\footnotesize{Internet Applications, ID1354}}
\chead{\footnotesize{Name of the Report}}
\rhead{}
\lfoot{}
\cfoot{\thepage\ (\pageref{LastPage})}
\rfoot{}

% Create title page
\title{}
\subtitle{Internet Applications, ID1354}
\author{Linus Berg Linus@Fenix.me.uk}
\date{\today}

\begin{document}

\maketitle

\section{Introduction}

The seminar consisted of learning JavaScript, jQuery, and implement Ajax request
into the recipes website.
\section{Literature Study}
I was already familiar with jQuery and Ajax requests, all previous seminars
already utilised jQuery and Ajax.

\section{Method}
As stated above, all previous seminars utilised Ajax and jQuery, therefor no
changes were made for this seminar. However my initial thought during development
was that Ajax is a more modern method for updating a webpage rather than conducting
POST/GET requests by redirection, and I chose to implment it as I found it easier.
\\\\
\noindent
In the previous seminars, a simple .php file with a script was developed,
as can be seen in \href{https://github.com/linus-dev/KTH-Projects/tree/master/ID1354/2}{seminar 2},
then each one was easily requested via Ajax and a response was easily managed.
This separated the PHP from HTML further, and made the program easier to manage.
\\\\
\noindent
During \href{https://github.com/linus-dev/KTH-Projects/tree/master/ID1354/3}{seminar 3}, when
a MVC framework had to be implemented, the files were switched to the Laravel routing, and utilised
the controllers instead to manage each Ajax request.

\begin{center}
    \begin{tabular}{  l | r }
    Tool & Choice \\ 
    \hline
    Editor & \textit{Vim}\\
    Version Control & \textit{Git - Github}\\
    Web Server & \textit{nginx}\\
    Database & \textit{PostgreSQL 10.5-1}\\
    PHP & \textit{7.2.10-1} \\
    Framework & \textit{Laravel 5.7}\\
    \end{tabular}
\end{center}

\section{Result}

\textbf{This section explains \textit{the result} of what you did.} \\

\noindent Present the solution. Explain your code and prove that it meets the requirements. It is very important to \textit{state each requirement that is met} and explain \textit{how you met it}. It is also important to include links to your code in your Git repository, user interface screen-shots, see Figure \ref{fig:ui}, and other figures to illustrate your reasoning. Also remember that these figures must be referenced in the text.

\section{Discussion}

\textbf{This section \textit{analysis} the result presented in the previous section.} \\

\noindent Summarize the requirements and \textit{clearly state which of them you have met}. What lessons have you learned and what problems did you face? How were the problems solved? Should you have done something differently?

\section{Comments About the Course}

Any comment(s) related to this course offering or to coming offerings is much appreciated. \textit{Please also tell approximately how much time you spent on the assignment}, including lectures and exercises. This is of great help for course evaluation.

\end{document}
